\documentclass[]{article}
\usepackage{lmodern}
\usepackage{amssymb,amsmath}
\usepackage{ifxetex,ifluatex}
\usepackage{fixltx2e} % provides \textsubscript
\ifnum 0\ifxetex 1\fi\ifluatex 1\fi=0 % if pdftex
  \usepackage[T1]{fontenc}
  \usepackage[utf8]{inputenc}
\else % if luatex or xelatex
  \ifxetex
    \usepackage{mathspec}
  \else
    \usepackage{fontspec}
  \fi
  \defaultfontfeatures{Ligatures=TeX,Scale=MatchLowercase}
\fi
% use upquote if available, for straight quotes in verbatim environments
\IfFileExists{upquote.sty}{\usepackage{upquote}}{}
% use microtype if available
\IfFileExists{microtype.sty}{%
\usepackage{microtype}
\UseMicrotypeSet[protrusion]{basicmath} % disable protrusion for tt fonts
}{}
\usepackage[margin=1in]{geometry}
\usepackage{hyperref}
\hypersetup{unicode=true,
            pdftitle={Statistical Computation HW1},
            pdfauthor={Jennifer Lieu},
            pdfborder={0 0 0},
            breaklinks=true}
\urlstyle{same}  % don't use monospace font for urls
\usepackage{color}
\usepackage{fancyvrb}
\newcommand{\VerbBar}{|}
\newcommand{\VERB}{\Verb[commandchars=\\\{\}]}
\DefineVerbatimEnvironment{Highlighting}{Verbatim}{commandchars=\\\{\}}
% Add ',fontsize=\small' for more characters per line
\usepackage{framed}
\definecolor{shadecolor}{RGB}{248,248,248}
\newenvironment{Shaded}{\begin{snugshade}}{\end{snugshade}}
\newcommand{\KeywordTok}[1]{\textcolor[rgb]{0.13,0.29,0.53}{\textbf{#1}}}
\newcommand{\DataTypeTok}[1]{\textcolor[rgb]{0.13,0.29,0.53}{#1}}
\newcommand{\DecValTok}[1]{\textcolor[rgb]{0.00,0.00,0.81}{#1}}
\newcommand{\BaseNTok}[1]{\textcolor[rgb]{0.00,0.00,0.81}{#1}}
\newcommand{\FloatTok}[1]{\textcolor[rgb]{0.00,0.00,0.81}{#1}}
\newcommand{\ConstantTok}[1]{\textcolor[rgb]{0.00,0.00,0.00}{#1}}
\newcommand{\CharTok}[1]{\textcolor[rgb]{0.31,0.60,0.02}{#1}}
\newcommand{\SpecialCharTok}[1]{\textcolor[rgb]{0.00,0.00,0.00}{#1}}
\newcommand{\StringTok}[1]{\textcolor[rgb]{0.31,0.60,0.02}{#1}}
\newcommand{\VerbatimStringTok}[1]{\textcolor[rgb]{0.31,0.60,0.02}{#1}}
\newcommand{\SpecialStringTok}[1]{\textcolor[rgb]{0.31,0.60,0.02}{#1}}
\newcommand{\ImportTok}[1]{#1}
\newcommand{\CommentTok}[1]{\textcolor[rgb]{0.56,0.35,0.01}{\textit{#1}}}
\newcommand{\DocumentationTok}[1]{\textcolor[rgb]{0.56,0.35,0.01}{\textbf{\textit{#1}}}}
\newcommand{\AnnotationTok}[1]{\textcolor[rgb]{0.56,0.35,0.01}{\textbf{\textit{#1}}}}
\newcommand{\CommentVarTok}[1]{\textcolor[rgb]{0.56,0.35,0.01}{\textbf{\textit{#1}}}}
\newcommand{\OtherTok}[1]{\textcolor[rgb]{0.56,0.35,0.01}{#1}}
\newcommand{\FunctionTok}[1]{\textcolor[rgb]{0.00,0.00,0.00}{#1}}
\newcommand{\VariableTok}[1]{\textcolor[rgb]{0.00,0.00,0.00}{#1}}
\newcommand{\ControlFlowTok}[1]{\textcolor[rgb]{0.13,0.29,0.53}{\textbf{#1}}}
\newcommand{\OperatorTok}[1]{\textcolor[rgb]{0.81,0.36,0.00}{\textbf{#1}}}
\newcommand{\BuiltInTok}[1]{#1}
\newcommand{\ExtensionTok}[1]{#1}
\newcommand{\PreprocessorTok}[1]{\textcolor[rgb]{0.56,0.35,0.01}{\textit{#1}}}
\newcommand{\AttributeTok}[1]{\textcolor[rgb]{0.77,0.63,0.00}{#1}}
\newcommand{\RegionMarkerTok}[1]{#1}
\newcommand{\InformationTok}[1]{\textcolor[rgb]{0.56,0.35,0.01}{\textbf{\textit{#1}}}}
\newcommand{\WarningTok}[1]{\textcolor[rgb]{0.56,0.35,0.01}{\textbf{\textit{#1}}}}
\newcommand{\AlertTok}[1]{\textcolor[rgb]{0.94,0.16,0.16}{#1}}
\newcommand{\ErrorTok}[1]{\textcolor[rgb]{0.64,0.00,0.00}{\textbf{#1}}}
\newcommand{\NormalTok}[1]{#1}
\usepackage{graphicx,grffile}
\makeatletter
\def\maxwidth{\ifdim\Gin@nat@width>\linewidth\linewidth\else\Gin@nat@width\fi}
\def\maxheight{\ifdim\Gin@nat@height>\textheight\textheight\else\Gin@nat@height\fi}
\makeatother
% Scale images if necessary, so that they will not overflow the page
% margins by default, and it is still possible to overwrite the defaults
% using explicit options in \includegraphics[width, height, ...]{}
\setkeys{Gin}{width=\maxwidth,height=\maxheight,keepaspectratio}
\IfFileExists{parskip.sty}{%
\usepackage{parskip}
}{% else
\setlength{\parindent}{0pt}
\setlength{\parskip}{6pt plus 2pt minus 1pt}
}
\setlength{\emergencystretch}{3em}  % prevent overfull lines
\providecommand{\tightlist}{%
  \setlength{\itemsep}{0pt}\setlength{\parskip}{0pt}}
\setcounter{secnumdepth}{0}
% Redefines (sub)paragraphs to behave more like sections
\ifx\paragraph\undefined\else
\let\oldparagraph\paragraph
\renewcommand{\paragraph}[1]{\oldparagraph{#1}\mbox{}}
\fi
\ifx\subparagraph\undefined\else
\let\oldsubparagraph\subparagraph
\renewcommand{\subparagraph}[1]{\oldsubparagraph{#1}\mbox{}}
\fi

%%% Use protect on footnotes to avoid problems with footnotes in titles
\let\rmarkdownfootnote\footnote%
\def\footnote{\protect\rmarkdownfootnote}

%%% Change title format to be more compact
\usepackage{titling}

% Create subtitle command for use in maketitle
\newcommand{\subtitle}[1]{
  \posttitle{
    \begin{center}\large#1\end{center}
    }
}

\setlength{\droptitle}{-2em}

  \title{Statistical Computation HW1}
    \pretitle{\vspace{\droptitle}\centering\huge}
  \posttitle{\par}
    \author{Jennifer Lieu}
    \preauthor{\centering\large\emph}
  \postauthor{\par}
      \predate{\centering\large\emph}
  \postdate{\par}
    \date{9/15/2018}


\begin{document}
\maketitle

Part 1: Importing Data into R i.

\begin{Shaded}
\begin{Highlighting}[]
\CommentTok{#set working directory and importing data}
\KeywordTok{setwd}\NormalTok{(}\StringTok{"/Users/jenniferlieu/Desktop/Stat_Comp_Data"}\NormalTok{)}
\NormalTok{titanic <-}\StringTok{ }\KeywordTok{read.table}\NormalTok{(}\StringTok{"Titanic.txt"}\NormalTok{, }\DataTypeTok{header=}\OtherTok{FALSE}\NormalTok{, }\DataTypeTok{as.is=}\OtherTok{TRUE}\NormalTok{)}
\end{Highlighting}
\end{Shaded}

\begin{enumerate}
\def\labelenumi{\roman{enumi}.}
\setcounter{enumi}{1}
\item
\end{enumerate}

\begin{Shaded}
\begin{Highlighting}[]
\CommentTok{#getting summary of data}
\KeywordTok{str}\NormalTok{(titanic)}
\end{Highlighting}
\end{Shaded}

\begin{verbatim}
## 'data.frame':    892 obs. of  12 variables:
##  $ V1 : chr  "PassengerId" "1" "2" "3" ...
##  $ V2 : chr  "Survived" "0" "1" "1" ...
##  $ V3 : chr  "Pclass" "3" "1" "3" ...
##  $ V4 : chr  "Name" "Braund, Mr. Owen Harris" "Cumings, Mrs. John Bradley (Florence Briggs Thayer)" "Heikkinen, Miss. Laina" ...
##  $ V5 : chr  "Sex" "male" "female" "female" ...
##  $ V6 : chr  "Age" "22" "38" "26" ...
##  $ V7 : chr  "SibSp" "1" "1" "0" ...
##  $ V8 : chr  "Parch" "0" "0" "0" ...
##  $ V9 : chr  "Ticket" "A/5 21171" "PC 17599" "STON/O2. 3101282" ...
##  $ V10: chr  "Fare" "7.25" "71.2833" "7.925" ...
##  $ V11: chr  "Cabin" "" "C85" "" ...
##  $ V12: chr  "Embarked" "S" "C" "S" ...
\end{verbatim}

From observing the output of the R code, we can tell that there are 12
variables, so 12 columns, and 892 obsercations, so 891 rows because we
have the first row being the labels. iii.

\begin{Shaded}
\begin{Highlighting}[]
\CommentTok{# initialize vector}
\NormalTok{v13 <-}\StringTok{ }\KeywordTok{c}\NormalTok{(}\DecValTok{1}\NormalTok{,}\DecValTok{2}\NormalTok{,}\DecValTok{3}\NormalTok{)}
\CommentTok{# Set label}
\NormalTok{v13[}\DecValTok{1}\NormalTok{] <-}\StringTok{ "Survived.Word"}
\CommentTok{#initialize parameter}
\NormalTok{i=}\DecValTok{2}
\CommentTok{#loop the 'survived' vector, and set v13 to survived or died depending on values in 'survived' vector.}
\ControlFlowTok{while}\NormalTok{(i }\OperatorTok{<}\StringTok{ }\DecValTok{893}\NormalTok{)\{}
  \ControlFlowTok{if}\NormalTok{(titanic[i,}\DecValTok{2}\NormalTok{] }\OperatorTok{==}\StringTok{"1"}\NormalTok{)\{}
\NormalTok{    v13[i]=}\StringTok{"survived"}
\NormalTok{  \}}
  \ControlFlowTok{else}\NormalTok{\{}
\NormalTok{    v13[i]=}\StringTok{"died"}
\NormalTok{  \}}
\NormalTok{  i=i}\OperatorTok{+}\DecValTok{1}
\NormalTok{\}}
\CommentTok{#adding column to dataset}
\NormalTok{titanic}\OperatorTok{$}\NormalTok{v13 <-}\StringTok{ }\NormalTok{v13}
\end{Highlighting}
\end{Shaded}

Part 2:Exploring the data i.

\begin{Shaded}
\begin{Highlighting}[]
\NormalTok{c <-}\StringTok{ }\KeywordTok{c}\NormalTok{(}\DecValTok{2}\NormalTok{,}\DecValTok{6}\NormalTok{,}\DecValTok{10}\NormalTok{)}
\NormalTok{datamatrix <-}\StringTok{ }\KeywordTok{data.matrix}\NormalTok{(titanic[}\DecValTok{2}\OperatorTok{:}\DecValTok{892}\NormalTok{,c])}
\KeywordTok{apply}\NormalTok{(datamatrix, }\DecValTok{2}\NormalTok{, mean)}
\end{Highlighting}
\end{Shaded}

\begin{verbatim}
##         V2         V6        V10 
##  0.3838384         NA 32.2042080
\end{verbatim}

The resulting mean for survived is the fraction of passengers who
survived the titanic. The reason why Age has a mean of `NA' is because
we do not have all the age values for NA in our dataset, and therefore
we cannot calculate the average Age of all the passengers on the voyage.
ii.


\end{document}
